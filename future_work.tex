\section{Future work}
\label{sec:future_work}

\paragraph{Relaxed memory model.}

The main limitation of our work is inherited from \Zoo: it relies on a sequentially consistent memory model whereas \OCamlFive has a relaxed memory model~\citep{DBLP:conf/pldi/DolanSM18}.
This simplification endangers the soundness of our specifications.
Transitioning to relaxed memory by merging \Zoo with \Cosmo~\citep{DBLP:journals/pacmpl/MevelJP20,DBLP:journals/pacmpl/MevelJ21} involves introducing memory views, which complicates specifications and invariants.

\paragraph{Language features.}

\Parabs suffers from the lack of a number of language features unsupported by \Zoo.
If functors were supported, we could make the \Parabs library completely modular.
If exceptions were supported, we could catch and re-raise exceptions in \ocamlinline{Pool} and \ocamlinline{Vertex}.
If algebraic effects were supported, we could get rid of evaluation contexts in \ocamlinline{Pool} and implement continuation-stealing.

\paragraph{Extensions.}

In the future, we would like to extend the library in several directions:
(1) develop the interface of futures, similarly to \Moonpool%
\footnote{
https://github.com/c-cube/moonpool/blob/main/src/core/fut.mli
}%
;
(2) support the different task types of \Taskflow, aiming at a more practical \ocamlinline{Vertex} interface.

\paragraph{Other designs.}

We could experiment other designs.
For instance, one of the two designs of \Moonpool relies on a bounded work-stealing deque combined with a master queue.
In the literature, many other scheduling strategies were proposed: continuation-stealing~\citep{DBLP:conf/ipps/SchmausPSHN21,DBLP:journals/tpds/WilliamsE25}, steal-half work-stealing~\citep{DBLP:conf/podc/HendlerS02}, split work-stealing~\citep{DBLP:conf/sc/DinanLSKN09,DBLP:journals/scheduling/RitoP22,DBLP:conf/europar/DijkP14,DBLP:conf/spaa/CustodioPR23,DBLP:conf/icpp/CartierDL21}, idempotent work-stealing~\citep{DBLP:conf/ppopp/MichaelVS09}.
