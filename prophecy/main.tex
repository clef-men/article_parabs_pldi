\def\model{%
  \l{\textcolor{color1}{model}}%
}
\def\full{%
  \l{\textcolor{color2}{full}}%
}
\def\snapshot{%
  \l{\textcolor{color3}{snapshot}}%
}
\def\lb{%
  \l{\textcolor{color4}{lb}}%
}


\section{Prophecy variables}
\label{sec:prophecy}

In \citeyear{DBLP:journals/pacmpl/JungLPRTDJ20}, \citeauthor{DBLP:journals/pacmpl/JungLPRTDJ20} introduced \emph{prophecy variables} in \Iris.
Essentially, prophecy variables --- or \emph{prophets}, as we will call them in this section --- can be used to predict the future of the program execution and reason about it.
They are key to handle \emph{future-dependent linearization points}~\citep{DBLP:journals/corr/DongolD14}: linearization points that may or may not occur at a given location in the code depending on a future observation.

In the program, prophecies take the form of two primitives: \c{Proph} and \c{Resolve}.
To reason about them in the logic, \citet{DBLP:journals/pacmpl/JungLPRTDJ20} proposed two abstraction layers, which we recall in \cref{sec:prophecy_primitive,sec:prophecy_typed}.
To verify the Chase-Lev work-stealing deque (see \cref{sec:chaselev}), we need to introduce two additional layers, presented in \cref{sec:prophecy_wise,sec:prophecy_multiplexed}.

\subsection{Primitive prophet}
\label{sec:prophecy_primitive}

\begin{figure}
\begin{mathpar}
    \infer[prophet-model-exclusive]{
        \model\ \zooProph\ \v{prophs}_1
      \\
        \model\ \zooProph\ \v{prophs}_2
    }{
      \iFalse
    }
  \and
    \iSpec[
      lab=wp-proph
    ]{
      \iTrue
    }{
      \zooExprProph
    }[
      \zooProph
    ]{
      \Exists \v{prophs}.
      \model\ \zooProph\ \v{prophs}
    }
  \and
    \infer[wp-resolve]{
        \iAtomic{\zooExpr}
      \\
        \l{to-val}\ \zooExpr = \None
      \\
        \model\ \zooProph\ \v{prophs}
      \\
        \iWp{
          \zooExpr
        }[
          \v{res}
        ]{
          \Forall \v{prophs'}. \\
          \v{prophs} = \cons{(\v{res}, \zooVal)}{\v{prophs'}} \iWand \\
          \model\ \zooProph\ \v{prophs'} \iWand \\
          \iPred\ \v{res}
        }
    }{
      \iWp{
        \zooExprResolve{\zooExpr}{\zooProph}{\zooVal}
      }{
        \iPred
      }
    }
\end{mathpar}
\caption{Reasoning rules for primitive prophets}
\label{fig:prophecy_primitive}
\end{figure}


The first layer consists of \emph{primitive prophets}~\refTheories{zoo/program_logic/wp.v}.
These prophets are primitive in the sense that they simply reflect the semantics of \c{Proph} and \c{Resolve} in the program logic.
The corresponding reasoning rules are given in \cref{fig:prophecy_primitive}.

The assertion $\model\ \zooProph\ \v{prophs}$ represents the exclusive ownership of the prophet with identifier \zooProph; \v{prophs} is the list of prophecies that must still be resolved.

\refTirName{wp-proph} says that \c{Proph} allocates a new prophet with some unknown prophecies to be resolved.
\refTirName{wp-resolve} says that $\c{Resolve}\ \zooExpr\ \zooProph\ \zooVal$ \emph{atomically} resolves the next prophecy of prophet \zooProph: we learn that the prophecies before resolution \v{prophs} is non-empty and its head is the pair $(\v{res}, \zooVal)$ where \v{res} is the evaluation of \zooExpr.


\subsection{Typed prophet}
\label{sec:prophecy_typed}

\begin{tirPrefix}{typed-prophet-}

\begin{figure}
\begin{mathpar}
    \infer[model-exclusive]{
        \model\ \zooProph\ \v{prophs}_1
      \\
        \model\ \zooProph\ \v{prophs}_2
    }{
      \iFalse
    }
  \and
    \iSpec[
      lab=proph-spec
    ]{
      \iTrue
    }{
      \zooExprProph
    }[
      \zooProph
    ]{
      \Exists \v{prophs}.
      \model\ \zooProph\ \v{prophs}
    }
  \and
    \infer[resolve-spec]{
        \iAtomic{\zooExpr}
      \\
        \l{to-val}\ \zooExpr = \None
      \\
        \zooVal = \recordGet{\v{prophet}}{to-val}\ \v{proph}
      \\
        \model\ \zooProph\ \v{prophs}
      \\
        \iWp{
          \zooExpr
        }[
          \zooValTwo
        ]{
          \Forall \v{prophs'}. \\
          \v{prophs} = \cons{\v{proph}}{\v{prophs'}} \iWand \\
          \model\ \zooProph\ \v{prophs'} \iWand \\
          \iPred\ \zooValTwo
        }
    }{
      \iWp{
        \zooExprResolve{\zooExpr}{\zooProph}{\zooVal}
      }{
        \iPred
      }
    }
\end{mathpar}
\caption{Reasoning rules for typed prophets}
\label{fig:prophecy_typed}
\end{figure}


The second layer consists of \emph{typed prophets}~\refTheories{zoo/program_logic/typed_prophet.v}.
They are very similar to primitive prophets except prophecies are now typed.
The corresponding reasoning rules, given in \cref{fig:prophecy_typed}, are essentially the same as before.
The prophet must provide a type $\tau$ along with two functions \l{of-val} and \l{to-val}.
\l{to-val} converts an inhabitant of $\tau$ to a value; \refTirName{resolve-spec} relies on it to enforce that the prophecies are well-typed.
\l{of-val} attempts to convert a value to $\tau$; it is used internally.
\l{of-val} and \l{to-val} must be compatible: $\l{of-val}\ (\l{to-val}\ \v{proph}) = \Some{\v{proph}}$.

\end{tirPrefix}


\subsection{Wise prophet}
\label{sec:prophecy_wise}

\begin{tirPrefix}{wise-prophet-}

\begin{figure}
\begin{mathpar}
    \iPersistent{(\full\ \iGname\ \v{prophs})}
  \and
    \iPersistent{(\snapshot\ \iGname\ \v{past}\ \v{prophs})}
  \and
    \iPersistent{(\lb\ \iGname\ \v{lb})}
  \\
    \infer[model-exclusive]{
        \model\ \zooProph\ \iGname_1\ \v{past}_1\ \v{prophs1}
      \\
        \model\ \zooProph\ \iGname_2\ \v{past}_2\ \v{prophs2}
    }{
      \iFalse
    }
  \and
    \infer[full-get]{
      \model\ \zooProph\ \iGname\ \v{past}\ \v{prophs}
    }{
      \full\ \iGname\ (\v{past} \dplus \v{prophs})
    }
  \and
    \infer[full-valid]{
        \model\ \zooProph\ \iGname\ \v{past}\ \v{prophs}_1
      \\
        \full\ \iGname\ \v{prophs}_2
    }{
      \v{prophs}_2 = \v{past} \dplus \v{prophs}_1
    }
  \and
    \infer[full-agree]{
        \full\ \iGname\ \v{prophs}_1
      \\
        \full\ \iGname\ \v{prophs}_2
    }{
      \v{prophs}_1 = \v{prophs}_2
    }
  \and
    \infer[snapshot-get]{
      \model\ \zooProph\ \iGname\ \v{past}\ \v{prophs}
    }{
      \snapshot\ \iGname\ \v{past}\ \v{prophs}
    }
  \and
    \infer[snapshot-valid]{
        \model\ \zooProph\ \iGname\ \v{past}_1\ \v{prophs}_1
      \\
        \snapshot\ \iGname\ \v{past}_2\ \v{prophs}_2
    }{
      \Exists \v{past}_3.
      \v{past}_1 = \v{past}_2 \dplus \v{past}_3 \iSep
      \v{prophs}_2 = \v{past}_3 \dplus \v{prophs}_1
    }
  \and
    \infer[lb-get]{
      \model\ \zooProph\ \iGname\ \v{past}\ \v{prophs}
    }{
      \lb\ \iGname\ \v{prophs}
    }
  \and
    \infer[lb-valid]{
      \model\ \zooProph\ \iGname\ \v{past}\ \v{prophs}
      \\
        \lb\ \iGname\ \v{lb}
    }{
      \Exists \v{past}_1\ \v{past}_2.
      \v{past} = \v{past}_1 \dplus \v{past}_2 \iSep
      \v{lb} = \v{past}_2 \dplus \v{prophs}
    }
  \\
    \iSpec[
      lab=proph-spec
    ]{
      \iTrue
    }{
      \zooExprProph
    }[
      \zooProph
    ]{
      \Exists \iGname\ \v{prophs}.
      \model\ \zooProph\ \iGname\ \nil\ \v{prophs}
    }
  \and
    \infer[resolve-spec]{
        \iAtomic{\zooExpr}
      \\
        \l{to-val}\ \zooExpr = \None
      \\
        \zooVal = \recordGet{\v{prophet}}{to-val}\ \v{proph}
      \\
        \model\ \zooProph\ \iGname\ \v{past}\ \v{prophs}
      \\
        \iWp{
          \zooExpr
        }[
          \zooValTwo
        ]{
          \Forall \v{prophs'}. \\
          \v{prophs} = \cons{\v{proph}}{\v{prophs'}} \iWand \\
          \model\ \zooProph\ \iGname\ (\v{past} \dplus [\v{proph}])\ \v{prophs'} \iWand \\
          \iPred\ \zooValTwo
        }
    }{
      \iWp{
        \zooExprResolve{\zooExpr}{\zooProph}{\zooVal}
      }{
        \iPred
      }
    }
\end{mathpar}
\caption{Reasoning rules for wise prophets}
\label{fig:prophecy_wise}
\end{figure}


The third layer consists of \emph{wise prophets}~\refTheories{zoo/program_logic/wise_prophet.v}.
These prophets \emph{remember} past prophecies.
The corresponding reasoning rules are given in \cref{fig:prophecy_wise}.

The exclusive assertion $\model\ \zooProph\ \iGname\ \v{past}\ \v{prophs}$ represents the ownership of the prophet with identifier \zooProph; \iGname is the logical name of the prophet; \v{past} is the list of prophecies resolved so far; \v{prophs} is the list of prophecies that must still be resolved.

The persistent assertion $\full\ \iGname\ \v{prophs}$ represents the list of all (resolved or not) prophecies associated to the prophet with name \iGname, as stated by \refTirName{full-valid}.

The persistent aassertion $\snapshot\ \iGname\ \v{past}\ \v{prophs}$ represents a snapshot of the state of the prophet with name \iGname at some point in the past.
\refTirName{snapshot-valid} allows to relate the current state of \model to the past state of \snapshot.

The persistent assertion $\lb\ \iGname\ \v{lb}$ represents a lower bound on the non-resolved prophecies of the prophet with name \iGname.
In particular, as stated by \refTirName{lb-valid}, the list of currently non-resolved prophecies carried by \model is always a suffix of \v{lb}.

\refTirName{resolve-spec} is the same as before, except we also update the list of resolved prophecies after resolution.

\end{tirPrefix}


\subsection{Multiplexed prophet}
\label{sec:prophecy_multiplexed}

\begin{tirPrefix}{wise-prophets-}

\begin{figure}
\begin{mathparpagebreakable}
    \iPersistent{(\full\ \iGname\ i\ \v{prophs})}
  \and
    \iPersistent{(\snapshot\ \iGname\ i\ \v{past}\ \v{prophs})}
  \and
    \iPersistent{(\lb\ \iGname\ i\ \v{lb})}
  \\
    \infer[model-exclusive]{
        \model\ \zooProph\ \iGname_1\ \v{pasts}_1\ \v{prophss1}
      \\
        \model\ \zooProph\ \iGname_2\ \v{pasts}_2\ \v{prophss2}
    }{
      \iFalse
    }
  \and
    \infer[full-get]{
      \model\ \zooProph\ \iGname\ \v{pasts}\ \v{prophss}
    }{
      \full\ \iGname\ i\ (\v{pasts}\ i \dplus \v{prophss}\ i)
    }
  \and
    \infer[full-valid]{
        \model\ \zooProph\ \iGname\ \v{pasts}\ \v{prophss}
      \\
        \full\ \iGname\ i\ \v{prophs}
    }{
      \v{prophs} = \v{pasts}\ i \dplus \v{prophss}\ i
    }
  \and
    \infer[full-agree]{
        \full\ \iGname\ i\ \v{prophs}_1
      \\
        \full\ \iGname\ i\ \v{prophs}_2
    }{
      \v{prophs}_1 = \v{prophs}_2
    }
  \and
    \infer[snapshot-get]{
      \model\ \zooProph\ \iGname\ \v{pasts}\ \v{prophss}
    }{
      \snapshot\ \iGname\ (\v{pasts}\ i)\ (\v{prophss}\ i)
    }
  \and
    \infer[snapshot-valid]{
        \model\ \zooProph\ \iGname\ \v{pasts}\ \v{prophss}
      \\
        \snapshot\ \iGname\ i\ (\v{pasts}\ i)\ (\v{prophss}\ i)
    }{
      \Exists \v{past'}.
      \v{pasts}\ i = \v{past} \dplus \v{past'} \iSep
      \v{prophs} = \v{past'} \dplus \v{prophss}\ i
    }
  \and
    \infer[lb-get]{
      \model\ \zooProph\ \iGname\ \v{pasts}\ \v{prophss}
    }{
      \lb\ \iGname\ i\ (\v{prophss}\ i)
    }
  \and
    \infer[lb-valid]{
      \model\ \zooProph\ \iGname\ \v{pasts}\ \v{prophss}
      \\
        \lb\ \iGname\ i\ \v{lb}
    }{
      \Exists \v{past}_1\ \v{past}_2.
      \v{pasts}\ i = \v{past}_1 \dplus \v{past}_2 \iSep
      \v{lb} = \v{past}_2 \dplus \v{prophss}\ i
    }
  \\
    \iSpec[
      lab=proph-spec
    ]{
      \iTrue
    }{
      \zooExprProph
    }[
      \zooProph
    ]{
      \Exists \iGname\ \v{prophss}.
      \model\ \zooProph\ \iGname\ (\Lambda \_. \nil)\ \v{prophss}
    }
  \and
    \infer[resolve-spec]{
        \iAtomic{\zooExpr}
      \\
        \l{to-val}\ \zooExpr = \None
      \\
        \zooVal = \recordGet{\v{prophet}}{to-val}\ \v{proph}
      \\
        \model\ \zooProph\ \iGname\ \v{pasts}\ \v{prophss}
      \\
        \iWp{
          \zooExpr
        }[
          \zooValTwo
        ]{
          \Forall \v{prophs}. \\
          \v{prophss}\ i = \cons{\v{proph}}{\v{prophs}} \iWand \\
          \model\ \zooProph\ \iGname\ (\mapAlter{(\cdot \dplus [\v{proph}])}{i}{\v{pasts}})\ (\mapInsert{i}{\v{prophs}}{\v{prophss}}) \iWand \\
          \iPred\ \zooValTwo
        }
    }{
      \iWp{
        \zooExprResolve{\zooExpr}{\zooProph}{(i, \zooVal)}
      }{
        \iPred
      }
    }
\end{mathparpagebreakable}
\caption{Reasoning rules for multiplexed prophets}
\label{fig:prophecy_multiplexed}
\end{figure}


The fourth layer consists of \emph{multiplexed prophets}~\refTheories{zoo/program_logic/wise_prophets.v}.
Essentially, they allow to combine different prophets, each operating at a fixed index.
They were made to handle the case when a single prophet is used to make independent predictions, as in \cref{sec:chaselev}.
The corresponding reasoning rules are given in \cref{fig:prophecy_multiplexed}.

The predicates and rules are basically the same as before, except that (1) \model now carries sequences of lists of prophecies --- one past and one future per index --- and (2) \full, \snapshot and \lb are parameterized with an index.

Importantly, the third argument provided to \c{Resolve} in \refTirName{resolve-spec} must be a pair of an index and a prophecy value.
Resolution happens only at the given index, meaning the prophecies at other indices are unchanged.

Note that we could generalize this abstraction to non-integer keys.
In other words, we could replace sequences with functions of type $X \rightarrow \tau$, where $\tau$ is the prophecy type, and indices with inhabitants of $X$.
In practice, however, we never needed such generalization.

\end{tirPrefix}

