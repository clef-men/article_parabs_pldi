\section{Benchmarks}
\label{app:benchmarks}

In this section, we present simple benchmarks to assess the performance of \Parabs relatively to \Domainslib~\citep{domainslib} and \Moonpool~\citep{moonpool} on simple workloads.
Benchmarking parallel schedulers is subtle and difficult; we have not tried here to validate and study experimentally all our implementation choices, or to cover the wide range of parallel workloads, but to validate a simple qualitative claim:

\begin{quotation}
  For CPU-bound tasks, \Parabs has comparable throughput to \Domainslib, a state-of-the-art scheduler used in production in the \OCamlFive library ecosystem.
\end{quotation}

In fact our results validate a stronger qualitative claim: the performance of \Parabs are equal or better than \Domainslib, with a 10\% speedup in some cases.

\subsection{Setting}

\subsubsection{Machine}

The benchmark results were produced on a 12-core AMD Ryzen 5 7640U machine, set at a fixed frequency of 2GHz.

\subsubsection{Parameters}

For each benchmark, we pick an input parameter that gives long-enough computation times on our test machine, typically between 200ms and 2s.
We use the \c{hyperfine} tool and run each benchmark ten time.
All benchmark were run with two parameters varying:
\begin{itemize}
  \item
    \c{DOMAINS}, the number of domains  used for computation;
  \item
    \c{CUTOFF}, representing an input size or chunk size below which a sequential baseline is used.
\end{itemize}

For each benchmark, we show:
\begin{itemize}
  \item
    per-cutoff results with a fixed value $\c{DOMAINS} = 6$, which should be enough to experience scaling issues while not suffering from CPU contention;
  \item
    per-domain results with a \c{CUTOFF} value that is chosen to work well for all implementations for this benchmark.
\end{itemize}

Remark: Large cutoff values tend to work well for benchmarks with homogeneous-enough tasks, as they effectively amortize the scheduling costs. The advantage of having schedulers that also perform well on small cutoffs are two-fold. First, this typically indicate that they will adapt to irregular tasks (but: our benchmarks do not perform an in-depth exploration of irregular workloads). Second, this can alleviate the burden of asking users to choose cutoff sizes (by widening the range of values that perform well), an activity which requires cumbersome hand-tuning and can limit performance portability.

\subsubsection{Scheduler implementations}

Each benchmark is written on top of a simple scheduler interface, for which the following implementations are provided:
\begin{itemize}
  \item
    \c{domainslib} uses the \Domainslib library;
  \item
    \c{parabs} uses our \Parabs library;
  \item
    \c{moonpool-fifo} uses the \Moonpool scheduler with a global FIFO queue of task;
  \item
    \c{moonpool-ws} uses the \Moonpool scheduler with a work-stealing pool of tasks,
    which is described as better for throughput
  \item
    \c{sequential} is a baseline implementation with no parallelism, all tasks run sequentially on a single domain.
\end{itemize}

We used the latest software versions currently available: \Domainslib 0.5.2, and \Moonpool 0.9.

\subsubsection{Benchmarks}

\paragraph{\c{fibonacci} \refBench{fibonacci/run.ml}.}

A parallel implementation of Fibonacci extended with a sequential cutoff: below the cutoff value, a sequential implementation is used.

\paragraph{\c{iota} \refBench{iota/run.ml}.}

This benchmark uses a parallel-for to write a default value in each cell of an array.
We expect significant variations due to the \c{CUTOFF} parameter.

\paragraph{\c{for\_irregular} \refBench{for_irregular/run.ml}.}

This benchmark uses a parallel-for loop with irregular per-element workload: as a first approximation, the $i$-th iteration computes $\l{fibonacci}\ i$; this cost grows exponentially in $i$, so the majority of computation work is concentrated on the largest loop indices.

\paragraph{\c{lu} \refBench{lu/run.ml}.}

This benchmark performs the LU factorization of a random matrix of floating-point values.
It consists in $O(N)$ repetitions of a parallel-for loop of $O(N)$ iterations, where each iteration performs $O(N)$ sequential work.

\paragraph{\c{matmul} \refBench{matmul/run.ml}.}

This benchmark computes matrix multiplication with a very simple parallelization strategy --- only the outer loop is parallelized.
In other word, there is a parallel-for loop with $O(N)$ iterations, where each iteration performs $O(N^2)$ sequential work work.

\subsection{Results}

\subsubsection{Pre-benchmarking expectations}

Our expectation before running the benchmarks is that \Parabs has the same performance as \Domainslib, and that they are both more efficient than \Moonpool (which uses a central pool of jobs instead of per-domain deques).

Because \Moonpool has a less optimized scheduler, we expect scheduling overhead to be an issue for small \c{CUTOFF} values.

On all schedulers, the performance for larger \c{CUTOFF} values should be good if the benchmark has homogeneous/regular tasks, and it should be worse if the benchmark has heterogeneous/irregular tasks.

\subsubsection{Per-benchmark results}

\def\scale{.45}
\def\myfigure#1{
  \begin{subfigure}[t]{.48\textwidth}
    \centering
    \includegraphics[scale=\scale]{benchmarks/data/#1-plot_cutoff.data.svg.pdf}
    \caption{\ocamlinline{#1}: varying \c{CUTOFF}}
    \label{fig:benchmarks_#1_cutoff}
  \end{subfigure}
  \hfill
  \begin{subfigure}[t]{.48\textwidth}
    \centering
    \includegraphics[scale=\scale]{benchmarks/data/#1-plot_domains.data.svg.pdf}
    \caption{\ocamlinline{#1}: varying \c{DOMAINS}}
    \label{fig:benchmarks_#1_domains}
  \end{subfigure}
}

\begin{figure}
\centering
\myfigure{fibonacci}
\par\bigskip
\myfigure{for_irregular}
\par\bigskip
\myfigure{iota}
\caption{Benchmarks (1/2)}
\label{fig:benchmarks}
\end{figure}

\begin{figure}
\ContinuedFloat
\centering
\myfigure{lu}
\par\bigskip
\myfigure{matmul}
\caption{Benchmarks (2/2)}
\label{fig:benchmarks}
\end{figure}


\cref{fig:benchmarks_1} and \cref{fig:benchmarks_2} contain the full results, with per-cutoff and per-domain plots for each benchmarks. Notice that while the per-domain plot always use linear axes, the per-cutoff plots often use logarithmic plot axes, to preserve readability when performance difference become very large for small cutoff values, and to express large ranges of possible cutoff choices.

\paragraph{\c{fibonacci}.}

In the per-cutoff results (logarithmic scale), we see that all schedulers start to behave badly when the \c{CUTOFF} becomes small enough, with exponentially-decreasing performance after a certain drop point.
For \Moonpool, performance drops around $\c{CUTOFF} = 20$. The FIFO and work-stealing variants have similar profiles, with work-stealing performing noticeably better.
For \Parabs and \Domainslib, performance drops around $\c{CUTOFF} = 12$. \Parabs performs noticeably better for small-enough cutoff values.
In fact, even for the sequential scheduler we observe a small performance drop: the task-using version creates closures and performs indirect calls, so it is noticeably slower (by a constant factor) than the version used below the sequential cutoff.

Note: we observe very large memory usage with \Moonpool at smaller cutoff values --- when computing $\l{fibonacci}\ 40$, attempting to run the benchmark with $\c{CUTOFF} = 5$ fails with out-of-memory errors on a machine with 32Gio of RAM.
This seems to come from the FIFO architecture which runs the oldest and thus biggest task first, and thus stores an exponential number of smaller tasks in the queue.

Per-domain results (linear scale): we studied per-domain performance on a $\c{CUTOFF}=25$ point where all implementations behave well. For this value we see that \c{parabs} and \c{domainslib} perform similarly, and both \c{moonpool} implementations are measurably slower.
Performance becomes very close for larger number of domains ($\c{DOMAINS} \geq 7$).

\paragraph{\c{for\_irregular}.}

This benchmark is designed to behave poorly with large \c{CUTOFF} values.
We indeed observe better noticeably performance with $\c{CUTOFF} = 1$ than with larger values, across all schedulers --- for example \c{domainslib} is 50\% slower with $\c{CUTOFF} = 8$.

In the per-cutoff results we observe that \c{parabs} performs best on this benchmark, then \c{domainslib}, then \c{moonpool}.

In the per-domain results (with $\c{CUTOFF} = 1$) we see that \c{parabs} performs noticeably better than the other implementations for relatively low domain counts, and they become comparable around $\c{DOMAINS} \geq 7$.

\paragraph{\c{iota}.}

Each iteration of parallel-for in \c{iota} is immediate, so as expected we observe a large sensitivity to the choice of \c{CUTOFF}, with \c{parabs} and \c{domainslib} performing much better than \c{moonpool} on smaller \c{CUTOFF} values.

In the per-domains result we see that \c{domainslib} and \c{parabs} have similar performance, noticeably better than the \c{moonpool} implementations.

\paragraph{\c{lu}.}

The performance is relatively stable over most choices of \c{CUTOFF}.
The per-domain results are similar across all benchmarks after controlling for the one-domain shift of \Moonpool.

One notable aspect of the result is that we observe a marked decline in performance, across all schedulers, when the number of domains becomes close to the number of available cores, around $\c{DOMAINS} \geq 10$. We believe that this comes from the high-allocation rate of this benchmark (10.2GiB/s) causing frequent minor collections, and thus stop-the-world pauses, with some domains temporarily suspended by the operating system. The allocations can be avoided in this benchmark by optimizing more agressively to eliminate float boxing, but this phenomenon is likely to occur for other high-allocation \OCaml programs so we chose to preserve it.

\paragraph{\c{matmul}.}

The performance is stable across a wide range of \c{CUTOFF} values.
The parallel-loop performs 500 iterations, so \c{CUTOFF} values closer to 500 prevent parallelization and bring performance closer to the sequential scheduler.

The per-domain performance is remarkably similar under all schedulers: our implementation of matrix multiplication has a coarse-grained parallelization strategy where the choice of scheduler makes no difference.

\subsubsection{Result summary}

Overall, \Parabs has the same qualitative performance as \Domainslib.
In fact it performs measurably better (around 10\% better for some domain values) on the benchmarks \c{fibonacci} and \c{for\_irregular}, which have irregular tasks; and it has qualitatively the same performance otherwise.
