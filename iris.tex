\section{\Iris arsenal}

\paragraph{Separation logic}

\Iris is a concurrent separation logic~\citep{DBLP:conf/csl/OHearnRY01,DBLP:conf/lics/Reynolds02,DBLP:journals/tcs/OHearn07} fully mechanized in the \Rocq proof assistant~\citep{DBLP:journals/pacmpl/KrebbersJ0TKTCD18}.
As such, it features basic connectives like separation conjunction $\iSep$ and separating implication $\iWand$.

\paragraph{Persistent assertions}

In \Iris, assertions are affine: using a resource consumes it, removes it from the proof context.
Some assertions, however, are \emph{persistent}.
Once a persistent assertion holds, it holds forever; using it does not consume it.
This enables \emph{duplication} ($\iProp \iEntails \iProp \iSep \iProp$) and \emph{sharing}.
In particular, pure (meta-level) assertions embedded into the logic are persistent.

Formally, persistence is defined in terms of the \emph{persistence modality}:
\[
  \iPersistent{\iProp}
  \eqdef
  \iProp \iEntails \iPersistently \iProp
\]
Informally, $\iPersistently \iProp$ means \iProp holds without asserting any exclusive ownership; in other words, it only expresses knowledge.
Naturally, $\iPersistently \iProp$ is persistent.

\paragraph{Ghost state}

One of the most important features of \Iris is its \emph{user-defined higher-order ghost state}, a very flexible form of ghost state.
Ghost updates, of the form $\iFupd \iProp$, allow updating the ghost state during the proof; they are purely logical, hence not visible in the program.

\paragraph{Sequential specification}

Sequential specifications take the form of Hoare triples:
\begin{mathpar}
    \iSpec{
      \iProp
    }{
      \zooExpr
    }{
      \iPred
    }
  \and
    \iSpec{
      \l{stack-model}\ t\ \zooVals
    }{
      \c{stack\_push}\ t\ \zooVal
    }[
      \c{()}
    ]{
      \l{stack-model}\ t\ (\cons{\zooVal}{\zooVals})
    }
\end{mathpar}
where $P$ is an \Iris assertion, $\zooExpr$ an expression and \iPred an \Iris predicate over values.

Informally, this triple says: if the precondition \iProp holds, we can safely execute $\zooExpr$ and, if the execution terminates, the returned value satisfies the postcondition \iPred.
It is a persistent resource, allowing executing $\zooExpr$ many times.

\paragraph{Weakest precondition}

Hoare triples are defined using the more primitive notion of \emph{weakest precondition} $\iWp{\zooExpr}{\iPred}$.
Informally, it says that: once only, we can execute $\zooExpr$ and, if the execution terminates, the returned value satisfies the postcondition \iPred.
Contrary to Hoare triples, it can depend on exclusive ownership and therefore is not persistent.

\paragraph{Atomic specification}

To specify concurrent operations, we use the notion of \emph{logical atomicity}~\citep{DBLP:conf/ecoop/PintoDG14}.
An operation is said to be logically atomic if it appears to take effect atomically at some point during its execution; this point is called the \emph{linearization point} of the operation.
\citeauthor{DBLP:journals/pacmpl/BirkedalDGJST21} showed that this notion implies \emph{linearizability}~\citep{DBLP:journals/toplas/HerlihyW90} in a sequentially consistent memory model.

In \Iris, logical atomicity takes the form of \emph{atomic specifications}:
\begin{mathpar}
    \iAspec{
      \iProp_\v{priv}
    }[
      \seq{x}
    ]{
      \iProp_\v{pub}
    }{
      \zooExpr
    }[
      \seq{y}
    ]{
      \iPropTwo
    }{
      \iPred
    }
  \and
    \iAspec{
      \l{stack-inv}\ t
    }[
      \zooVals
    ]{
      \l{stack-model}\ t\ \zooVals
    }{
      \c{stack\_push}\ t\ \zooVal
    }{
      \l{stack-model}\ t\ (\cons{\zooVal}{\zooVals})
    }[
      \c{()}
    ]{
      \iTrue
    }
\end{mathpar}
$\iProp_\v{priv}$ and \iPred are standard \emph{private} pre- and postcondition for the user of the specification, similarly to Hoare triples.
$\iProp_\v{pub}$ and \iPropTwo are \emph{public} pre- and postcondition; they specify the linearization point of the operation.
Quantifiers $\seq{x}$ represent the \emph{demonic nature} of $\iProp_\v{pub}$: the exact state at the linearization point, given by $\iProp_\v{pub}$, is unknown until it happens.
Quantifiers $\seq{y}$ represent the \emph{angelic nature} of \iPropTwo: at the linearization point, the operation can choose how to instantiate the new state \iPropTwo.

In sum, the atomic specification says: if the private precondition $\iProp_\v{priv}$ holds, we can safely execute $\zooExpr$ and, if the execution terminates, (1) the returned value satisfies the private postcondition \iPred and (2) at some point during the execution, the state was atomically updated from $\iProp_\v{pub}$ to \iPropTwo.
