\section{Work-stealing deques}

At the first level, \ocamlinline{Ws_deques}~\refLib{zoo_parabs/ws_deques.ml} provides a generic interface for a set of work-stealing deques, abstracting over the underlying scheduling strategy.
It currently has two realizations: \ocamlinline{Ws_deques_public} (\cref{sec:parabs_ws_deques_public}) and \ocamlinline{Ws_deques_private} (\cref{sec:parabs_ws_deques_private}).

\subsection{Public deques}
\label{sec:parabs_ws_deques_public}

The first realization, \ocamlinline{Ws_deques_public}~\refLibTheoriesFile{zoo_parabs/ws_deques_public}, implements the standard work-stealing strategy with \emph{public deques}.
More precisely, it simply relies on a shared array of Chase-Lev work-stealing deques, as implemented in \Saturn (see \cref{sec:chaselev}).
These deques are public in the sense that both their owner and the thieves can access it directly --- which requires synchronization.

\subsection{Private deques}
\label{sec:parabs_ws_deques_private}

The second realization, \ocamlinline{Ws_deques_private}~\refLibTheoriesFile{zoo_parabs/ws_deques_private}, implements the \emph{receiver-initiated} work-stealing algorithm proposed by \citet{DBLP:conf/ppopp/AcarCR13}%
\footnote{
They also propose a \emph{sender-initiated} algorithm that we have not implemented.
}%
.
Their idea is to reduce synchronization costs in the fast path of local (owner-only) operations by essentially introducing an indirection.
They show that this work-stealing strategy performs well for \emph{fine-grained} parallel programs, \ie when task sizes are small, especially irregular graph computations.

Instead of stealing directly from public deques, thieves follow a protocol:
(1) having selected a victim, a thief attempts to send a request by atomically updating the \emph{request cell} of the victim;
(2) if the update fails, the thief starts over with another victim, otherwise it awaits a response by repeatedly checking its \emph{response cell};
(3) if the response is negative, the thief starts over, otherwise it returns the task transferred by the victim.

Symmetrically, busy domains regularly poll their request cell and respond accordingly through response cells.
Crucially, tasks are stored in private, non-concurrent deques that are only accessed by their owner.
In addition, each domain has a \emph{status cell} indicating whether it is (1) blocked, meaning it has no task to share, or (2) non-blocked, meaning it may have tasks to share; before sending a request, thieves check that their victim is non-blocked.
