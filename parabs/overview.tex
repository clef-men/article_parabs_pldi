\section{Overview}

\section{Overview}

\section{Overview}

\section{Overview}

\input{parabs/figures/overview}

\Cref{fig:parabs_overview} gives an overview of \Parabs; solid edges represent module dependencies while dashed edges represent interface implementations.
Essentially, the library is made of four abstraction levels built on top of each other: \ocamlinline{Ws_deques}, \ocamlinline{Ws_hub}, \ocamlinline{Pool} and \ocamlinline{Future} / \ocamlinline{Vertex}.

The \ocamlinline{Pool} module provides a task scheduler; internally, it maintains a pool of domains.
Its design is inspired by \Domainslib, \Taskflow~\citep{DBLP:journals/tpds/HuangLLL22} and \Moonpool~\citep{moonpool}.
As of today, it supports three scheduling strategies:
(1) standard randomized work-stealing~\citep{DBLP:journals/jacm/BlumofeL99} with public deques (as presented in \cref{sec:chaselev}),
(2) randomized work-stealing with private deques~\citep{DBLP:conf/ppopp/AcarCR13},
(3) a simple ``first-in first-out'' strategy with one shared queue.
In addition, it should be possible to implement other scheduling strategies (see \cref{sec:future_work}), \eg work sharing.

On top of \ocamlinline{Pool}, the \ocamlinline{Vertex} module provides a \emph{task graph} abstraction.
More precisely, it is an implementation of \emph{DAG-calculus}~\citep{DBLP:conf/icfp/AcarCRS16} --- we present it in \cref{sec:parabs_vertex}.

It is worth noting that the three upper levels implemented on top of \ocamlinline{Ws_deques} should be \OCaml functors.
Unfortunately, \ZooLang does not currently support functors; therefore, only one branch of the tree of \cref{fig:parabs_overview} is active at a time.


\Cref{fig:parabs_overview} gives an overview of \Parabs; solid edges represent module dependencies while dashed edges represent interface implementations.
Essentially, the library is made of four abstraction levels built on top of each other: \ocamlinline{Ws_deques}, \ocamlinline{Ws_hub}, \ocamlinline{Pool} and \ocamlinline{Future} / \ocamlinline{Vertex}.

The \ocamlinline{Pool} module provides a task scheduler; internally, it maintains a pool of domains.
Its design is inspired by \Domainslib, \Taskflow~\citep{DBLP:journals/tpds/HuangLLL22} and \Moonpool~\citep{moonpool}.
As of today, it supports three scheduling strategies:
(1) standard randomized work-stealing~\citep{DBLP:journals/jacm/BlumofeL99} with public deques (as presented in \cref{sec:chaselev}),
(2) randomized work-stealing with private deques~\citep{DBLP:conf/ppopp/AcarCR13},
(3) a simple ``first-in first-out'' strategy with one shared queue.
In addition, it should be possible to implement other scheduling strategies (see \cref{sec:future_work}), \eg work sharing.

On top of \ocamlinline{Pool}, the \ocamlinline{Vertex} module provides a \emph{task graph} abstraction.
More precisely, it is an implementation of \emph{DAG-calculus}~\citep{DBLP:conf/icfp/AcarCRS16} --- we present it in \cref{sec:parabs_vertex}.

It is worth noting that the three upper levels implemented on top of \ocamlinline{Ws_deques} should be \OCaml functors.
Unfortunately, \ZooLang does not currently support functors; therefore, only one branch of the tree of \cref{fig:parabs_overview} is active at a time.


\Cref{fig:parabs_overview} gives an overview of \Parabs; solid edges represent module dependencies while dashed edges represent interface implementations.
Essentially, the library is made of four abstraction levels built on top of each other: \ocamlinline{Ws_deques}, \ocamlinline{Ws_hub}, \ocamlinline{Pool} and \ocamlinline{Future} / \ocamlinline{Vertex}.

The \ocamlinline{Pool} module provides a task scheduler; internally, it maintains a pool of domains.
Its design is inspired by \Domainslib, \Taskflow~\citep{DBLP:journals/tpds/HuangLLL22} and \Moonpool~\citep{moonpool}.
As of today, it supports three scheduling strategies:
(1) standard randomized work-stealing~\citep{DBLP:journals/jacm/BlumofeL99} with public deques (as presented in \cref{sec:chaselev}),
(2) randomized work-stealing with private deques~\citep{DBLP:conf/ppopp/AcarCR13},
(3) a simple ``first-in first-out'' strategy with one shared queue.
In addition, it should be possible to implement other scheduling strategies (see \cref{sec:future_work}), \eg work sharing.

On top of \ocamlinline{Pool}, the \ocamlinline{Vertex} module provides a \emph{task graph} abstraction.
More precisely, it is an implementation of \emph{DAG-calculus}~\citep{DBLP:conf/icfp/AcarCRS16} --- we present it in \cref{sec:parabs_vertex}.

It is worth noting that the three upper levels implemented on top of \ocamlinline{Ws_deques} should be \OCaml functors.
Unfortunately, \ZooLang does not currently support functors; therefore, only one branch of the tree of \cref{fig:parabs_overview} is active at a time.


\Cref{fig:parabs_overview} gives an overview of \Parabs; solid edges represent module dependencies while dashed edges represent interface implementations.
Essentially, the library is made of four abstraction levels built on top of each other: \ocamlinline{Ws_deques}, \ocamlinline{Ws_hub}, \ocamlinline{Pool} and \ocamlinline{Future} / \ocamlinline{Vertex}.

The \ocamlinline{Pool} module provides a task scheduler; internally, it maintains a pool of domains.
Its design is inspired by \Domainslib, \Taskflow~\citep{DBLP:journals/tpds/HuangLLL22} and \Moonpool~\citep{moonpool}.
As of today, it supports three scheduling strategies:
(1) standard randomized work-stealing~\citep{DBLP:journals/jacm/BlumofeL99} with public deques (as presented in \cref{sec:chaselev}),
(2) randomized work-stealing with private deques~\citep{DBLP:conf/ppopp/AcarCR13},
(3) a simple ``first-in first-out'' strategy with one shared queue.
In addition, it should be possible to implement other scheduling strategies (see \cref{sec:future_work}), \eg work sharing.

On top of \ocamlinline{Pool}, the \ocamlinline{Vertex} module provides a \emph{task graph} abstraction.
More precisely, it is an implementation of \emph{DAG-calculus}~\citep{DBLP:conf/icfp/AcarCRS16} --- we present it in \cref{sec:parabs_vertex}.

It is worth noting that the three upper levels implemented on top of \ocamlinline{Ws_deques} should be \OCaml functors.
Unfortunately, \ZooLang does not currently support functors; therefore, only one branch of the tree of \cref{fig:parabs_overview} is active at a time.
