\section{Pool}
\label{sec:parabs_pool}

At the third level, \ocamlinline{Pool}~\refLibTheoriesFile{zoo_parabs/pool} implements a task scheduler on top of a given realization of \ocamlinline{Ws_hub}.
It offers essentially the same functionalities as \Domainslib with a few notable differences.
(1) Exceptions raised by tasks are not caught and therefore not re-raised properly by the scheduler since \ZooLang does not currently support them.
(2) Since \ZooLang does not support algebraic effects~\citep{DBLP:conf/pldi/Sivaramakrishnan21} neither, the interface is slightly more involved (see \emph{execution contexts} in \cref{sec:parabs_pool_spec}).

Moreover, this limitation imposes a \emph{child-stealing} strategy, as opposed to a \emph{continuation-stealing} strategy that would require capturing the continuation of a computation.

Also, this makes it difficult to implement a \c{yield} operation%
\footnote{
\Domainslib does not currently provide a \c{yield} operation but it can be easily implemented.
}%
, \ie an operation that yields control to the scheduler, letting it reschedule the current task later.

\subsection{Specification}
\label{sec:parabs_pool_spec}

\begin{tirPrefix}{pool-}

\begin{figure}
\begin{mathpar}
    \iPersistent{(\inv\ t\ \v{sz})}
  \and
    \iPersistent{(\obligation\ t\ \iProp)}
  \and
    \iPersistent{(\finished\ t)}
  \\
    \infer[inv-agree]{
        \inv\ t\ \v{sz}_1
      \\
        \inv\ t\ \v{sz}_2
    }{
      \v{sz}_1 = \v{sz}_2
    }
  \and
    \infer[obligation-finished]{
        \obligation\ t\ \iProp
      \\
        \finished\ t
    }{
      \iLater \iPersistently \iProp
    }
  \\
    \iSpec[
      lab=create-spec
    ]{
      0 \leq \v{sz}
    }{
      \c{create}\ \v{sz}
    }[
      t
    ]{
      \inv\ t\ \v{sz} \iSep \\
      \model\ t
    }
  \and
    \iSpec[
      lab=run-spec
    ]{
      \model\ t \iSep \\
      \begin{inboxIndent}
        \Forall \v{ctx}\ \v{scope}. \\
        \context\ t\ \v{ctx}\ \v{scope} \iWand \\
        \iWp{
          \v{task}\ \v{ctx}
        }[
          \zooVal
        ]{
          \context\ t\ \v{ctx}\ \v{scope} \iSep
          \iPredTwo\ \zooVal
        }
      \end{inboxIndent}
    }{
      \c{run}\ t\ \v{task}
    }[
      \zooVal
    ]{
      \model\ t \iSep
      \iPredTwo\ \zooVal
    }
  \and
    \iSpec[
      lab=kill-spec
    ]{
      \model\ t
    }{
      \c{kill}\ t
    }[
      \c{()}
    ]{
      \finished\ t
    }
  \and
    \iSpec[
      lab=size-spec
    ]{
      \inv\ t\ \v{sz} \iSep \\
      \context\ t\ \v{ctx}\ \v{scope}
    }{
      \c{size}\ \v{ctx}
    }[
      \v{res}
    ]{
      \v{res} = \v{sz} \iSep \\
      \context\ t\ \v{ctx}\ \v{scope}
    }
  \and
    \iSpec[
      lab=async-spec
    ]{
      \context\ t\ \v{ctx}\ \v{scope} \iSep \\
      \begin{inboxIndent}
        \Forall \v{ctx}\ \v{scope}. \\
        \context\ t\ \v{ctx}\ \v{scope} \iWand \\
        \iWp{
          \v{task}\ \v{ctx}
        }[
          \_
        ]{
          \context\ t\ \v{ctx}\ \v{scope} \iSep
          \iLater \iPersistently \iProp
        }
      \end{inboxIndent}
    }{
      \c{async}\ \v{ctx}\ \v{task}
    }[
      \c{()}
    ]{
      \context\ t\ \v{ctx}\ \v{scope} \iSep \\
      \obligation\ t\ \iProp
    }
  \and
    \iSpec[
      lab=wait-until-spec
    ]{
      \context\ t\ \v{ctx}\ \v{scope} \iSep \\
      \iTriple{
        \iTrue
      }{
        \v{pred}\ \c{()}
      }[
        \zooBool
      ]{
        \myifOneline{\zooBool}{\iProp}{\iTrue}
      }
    }{
      \c{wait\_until}\ \v{ctx}\ \v{pred}
    }[
      \c{()}
    ]{
      \context\ t\ \v{ctx}\ \v{scope} \iSep
      \iProp
    }
\end{mathpar}
\caption{\ocamlinline{Pool}: Specification}
\label{fig:parabs_pool_spec}
\end{figure}


The specification is given in \cref{fig:parabs_pool_spec}.
It features five predicates: \inv, \model, \context, \finished and \obligation.

The persistent assertion $\inv\ t\ v{sz}$ represents the knowledge that $t$ is a valid scheduler; \v{sz} is the number of worker domains.
It is returned by \c{create} (\refTirName{create-spec}) and required only by \c{size} (\refTirName{size-spec}).
Its only purpose is to record the immutable characteristics of the scheduler.

The assertion $\model\ t$ represents the ownership of scheduler $t$.
It is returned by \c{create} (\refTirName{create-spec}) and required by external operations (\refTirName{run-spec}, \refTirName{kill-spec}).
For example, $\c{run}\ t\ \v{task}$ submits \v{task} to scheduler $t$; it returns both \model and the output predicate of \v{task}.

The assertion $\context\ t\ \v{ctx}\ \v{scope}$ represents the ownership of \emph{execution context} \v{ctx} attached to scheduler $t$; \v{scope} is a purely logical parameter connecting input and output \context, which is necessary in the proof.
Any task execution happens under such a context (\refTirName{run-spec}, \refTirName{async-spec}, \refTirName{wait-until-spec}).
In particular, all internal operations require and return \context.
For example, $\c{async}\ \v{ctx}\ \v{task}$ submits \v{task} asynchronously while executing under context \v{ctx}; \v{task} must be shown to execute safely under any context attached to the same scheduler (\refTirName{async-spec}).

The persistent assertion $\finished\ t$ represents the knowledge that scheduler $t$ has finished, meaning all submitted tasks were executed.
It can be obtained by calling \c{kill} (\refTirName{kill-spec}).

The persistent assertion $\obligation\ t\ \iProp$ represents a proof obligation attached to scheduler $t$.
It allows retrieving \iProp once $t$ has finished executing (\refTirName{obligation-finished}).
Obligations are obtained by submitting tasks through \c{async} (\refTirName{async-spec}).

\end{tirPrefix}

\subsection{Implementation}
\label{sec:parabs_pool_impl}

\paragraph{Worker domains.}

The implementation relies on a pool of worker domains and a work-stealing hub.
Each worker runs the following loop:
(1) get a task using \ocamlinline{Ws_hub.pop_steal};
(2) if it fails, the scheduler has been killed and so the worker stops, otherwise execute the task in the context of the current worker;
(3) start over.

\paragraph{Blocking.}

Care must be taken to block and unblock work-stealing deques properly.
When the scheduler is killed, it is crucial that workers block their deque before stopping; otherwise, the scheduler may never terminate because of a running worker waiting forever for a response from a stopped but unblocked worker.
Also, the main domain, from which tasks can be submitted externally through \c{run}, must unblock when it is executing tasks and block when it is not.

\paragraph{Shutdown.}

In \Domainslib, scheduler shutdown consists in submitting special tasks through the main domain; when a worker finds such a task, it quickly stops.
However, this simple mechanism has at least two drawbacks:
(1) it introduces an indirection for every regular task, which may be expensive;
(2) it works well under standard work-stealing but is more difficult to implement under other scheduling strategies, especially work-stealing with private deques (see \cref{sec:parabs_ws_deques_private}).
Consequently, we use an alternative mechanism implemented at the level of \ocamlinline{Ws_hub}: a shared flag, regularly checked in \ocamlinline{Ws_hub.steal} and \ocamlinline{Ws_hub.pop_steal}, is set when the scheduler is killed.
